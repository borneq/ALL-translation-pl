\subsection{Dwuetapowa analiza składniowa ALL(*)}
SLL jest słabsze, ale szybsze niż LL.
Z faktu iż wykazano, że większość decyzji jest SLL w praktyce, wydaje się
sensowne aby wykonać próbę analizy składniowej wszystkich danych wejściowych
„tylko w trybie SLL”, co jest pierwszym etapem dwuetapowej analizy składniowej algorytmu ALL(*).
Jeśli jednak, tryb SLL wykryje błąd składniowy, wówczas albo jest to słabość SLL
albo prawdziwy błąd składniowy, tak więc musimy ponownie rozpatrzyć wszystkie
dane wejściowe używający zoptymalizowanego trybu LL, który jest drugim etapem.
Ta nieintuicyjna strategia, która potencjalnie parsuje cały strumień wejściowy dwukrotnie,
może dramatycznie zwiększyć prędkość w porównaniu do jednoetapowego zoptymalizowanego trybu LL.
Na przykład, dwu-etapowa analiza składniowa z gramatyką Java (Część 7) jest 8x szybsza
niż jednoetapowy zoptymalizowany tryb LL do przeprowadzania analizy składniowej korpus 123M.
Dwuetapowa strategia opiera się na fakcie, że SLL zachowuje się jak LL
lub otrzymujemy błąd składniowy (Twierdzenie 6.5).
Dla nieprawidłowych zdań, \textbf{nie występuje żadne wyprowadzenie dla danych
wejściowych} niezależnie od tego jak parser wybiera produkcje.
Dla prawidłowych zdań, SLL wybiera produkcje podobnie jak LL albo dobiera produkcję,
która ostatecznie prowadzi do błędu składniowego (LL wykazuje niewykonalny wybór).
Nawet w obecności niejednoznaczności, SLL często rozwiązuje konflikty, podobnie jak LL.
Na przykład, pomimo kilku niejednoznaczności w naszej gramatyce Javy, tryb SLL
poprawnie przeprowadza analizę składniową wszystkich danych wejściowych,
bez doznania niepowodzenia w LL. Niemniej jednak,
drugi etap (LL) musi pozostawać do zapewnienia poprawności.
