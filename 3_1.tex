\subsection{Przewidywania wrażliwe na stos wywołań}
Parsery nie zawsze mogą opierać się na lookahead DFA w podejmowaniu właściwej decyzji.
Aby obsłużyć wszystkie gramatyki nie lewostronnie rekurencyjne,
przewidywanie ALL(*) musi od czasu do czasu rozważyć stos wywołań parsera
dostępnego na początku przewidywania (oznaczone jako 0 w Części 5).
Aby przedstawić potrzebę przewidywań zależnego od stosu, należy wziąć pod uwagę
że przewidywania podczas rozpoznawania metody Javy mogą zależeć od tego
czy metoda została zdefiniowana w ramach definicji klasy lub interfejsu.
(metody interfejsu Java nie mogą posiadać ciała).
Oto przykład uproszczonej gramatyki, która wykazuje decyzję zależną od stosu dla symbolu nieterminalnego A:
\par
\( S \rightarrow xB|yC  \;\;\; B \rightarrow Aa \;\;\; C \rightarrow Aba \;\;\; A \rightarrow b|\varepsilon\)
\\
Bez stosu analizatora, żadna ilość lookahead nie może jednoznacznie rozróżnić
produkcji A. Lookahead ba przewiduje \(A \rightarrow b\) gdy B wywołuje A,
natomiast przewiduje \(A \rightarrow \varepsilon \) gdy C wywołuje A.
W przypadku gdy przewidywanie ignoruje stos wywołań analizatora,
występuje konflikt przewidywania wobec ba.
\par
Parsery, które ignorują stos wywołań analizatora dla przewidywania nazywamy
Silnymi (Strong) LL (SLL). Parsery rekurencyjne programiści tworzą ręcznie
w klasie SLL. Zazwyczaj, literatura odnosi się do SLL jako LL,
ale my rozróżniamy pojęcia, ponieważ „rzeczywiste” LL jest potrzebne aby
obsłużyć wszystkie gramatyki. Powyższa gramatyka to LL(2), a nie SLL(k) dla
jakiegokolwiek k, jednak duplikując A otrzymamy gramatykę SLL (2).
\par
Tworzenie różnych lookahead DFA dla każdego możliwego stosu wywołań
analizatora nie jest możliwe, ponieważ liczba permutacji stosu jest
wykładnicza w funkcji głębokości stosu.
Zamiast tego możemy skorzystać z faktu, iż większość decyzji nie jest zależna
od stosu i utworzyć lookahead DFA ignorując stos wywołań analizatora.
Jeśli symulacja ATN SLL znajduje konflikt przewidywania (Część 5.3),
to nie możemy być pewni czy zwrot lookahead nie jest jednoznaczny
albo zależny od stosu. 
W tym przypadku, adaptivePredict musi ponownie zbadać lookahead używając stosu parsera $\gamma_0$.
Ten hybrydowy lub zoptymalizowany tryb LL zwiększa wydajność poprzez zapisywaniu
w pamięci podręcznej wyników przewidywania niezależnych od stosu w lookahead DFA,
gdy jest to możliwe, przy zachowaniu pełnej siły przewidywania zależnej od siły stosu.
Zoptymalizowany tryb LL stosuje się na podstawie decyzji,
ale dwustopniowa analiza składniowa, opisana w dalszej części,
często może całkowicie uniknąć symulacji LL.
(Od tej pory używamy SLL, aby wskazać analizę składniową niezależną od stosu
i LL do wskazania zależności od stosu).
