\subsection{Dynamiczna analiza gramatyczna}
W niniejszym artykule przedstawiliśmy ALL(*) czyli adaptacyjne LL(*) - parsery
które łączą w sobie prostotę deterministycznych zstępujących parserów z
siłą mechanizmów GLR odnośnie analizy decyzyjnej.
W szczególności, analiza składniowa LL zawiesza działanie w każdym punkcie
przewidywania decyzji (symbol nieterminalny), a następnie zostaje wznowiona
po wyborze odpowiedniej produkcji do rozwinięcia przez mechanizm przewidywania.
Kluczową innowacją jest nakierowanie analizy gramatycznej na czas analizy,
nie jest tu potrzebna żadna statyczna analiza gramatyczna.
Wybór ten pozwoli nam uniknąć nierozstrzygalności statycznej LL(*) w analizie
gramatycznej i umożliwi wygenerowanie odpowiednich parserów (Twierdzenie 6.1)
dla nie-lewostronnej rekursywnej bezkontekstowej gramatyki (CFG).
Podczas gdy analiza statyczna musi uwzględniać wszystkie możliwe sekwencje
wejściowe, analiza dynamiczna rozważa tylko skończony zbiór sekwencji
wejściowych w rzeczywistości widoczny.
\par
Ideą mechanizmu przewidywania ALL(*) jest uruchomienie subparserów w punkcie
decyzyjnym, jeden na alternatywą produkcję.
Subparsery działają pseudo-równolegle odkrywając wszystkie możliwe ścieżki.
„Wymierają” te subparsery których ścieżki nie dopasowują pozostałych
danych wejściowych.
Subparsery przechodzą przez dane wejściowe równym krokiem tak że analiza może
zidentyfikować jedyny ocalały subparser na minimalnej głębokości podglądu
symboli (lookahead) który jednoznacznie przewiduje produkcję.
Jeśli wiele subparserów łączy się razem albo dochodzi do końca pliku, predyktor
ogłasza niejednoznaczność i rozwiązuje to na korzyść produkcji o najniższym
numerze powiązanej z ocalałym subparserem.
(Produkcje są numerowane co w automatyczny sposób określa pierwszeństwo
rozwiązując niejednoznaczności, niczym PEG; Bizon również rozwiązuje konflikty
wybierając produkcję określoną jako pierwszą).
Programiści mogą również wstawiać semantyczne predyktory [22], aby wybrać
między niejednoznacznymi interpretacjami.
\par
Parsery ALL(*) zapamiętują wyniki analizy, stopniowo i dynamicznie budując
cache automatów DFA, która mapuje frazy lookahead do przewidzianych produkcji.
(Wykorzystujemy termin analiza w tym sensie że analiza ALL(*) przynosi
lookahead DFA jak statyczna analiza LL(*)).
Analizator może tworzyć przyszłe przewidywania w takiej samej decyzji
analizatora i frazy lookahead poprzez zapoznanie się z cache.
Nieznane frazy wejściowe uruchamiają mechanizm analizy gramatycznej,
jednocześnie przewidując alternatywę i uaktualniając DFA.
DFA są odpowiednie dla wyników przewidywania, pomimo faktu że język lookahead
przy danej decyzji zazwyczaj tworzy język bezkontekstowy.
Analiza dynamiczna musi tylko uwzględniać skończone podzbiory języka
bezkontekstowego napotkane podczas przeprowadzania analizy składniowej,
a jakikolwiek skończony zbiór jest regularny.
\par
Aby uniknąć wykładniczej natury niedeterministycznych subparserów,
przewidywanie wykorzystuje graph-structured stack (GSS) [25] w celu uniknięcia
zbędnych obliczeń.
GLR wykorzystuje zasadniczo tą samą strategię, jednakże ALL(*) przewiduje tylko
produkcje z takimi subparserami, natomiast GLR faktycznie przeprowadza analizę
składniową z nimi.
Wskutek tego, GLR muszą wkładać symbole terminalne na GSS, czego ALL(*) nie robi.
\par
Parsery ALL(*) dopasowują symbole terminalne i rozwijają nieterminalne z
prostotą LL, jednakże posiadają teoretyczną złożoność czasową O($n^4$)
(Twierdzenie 6.3), ponieważ w najgorszym przypadku, parser musi przewidywać
przy każdym symbolu wejściowym, a każde przewidywanie musi analizować pozostałe
dane wejściowe; badanie symbolu wejściowego może być kosztowne O($n^2$).
O($n^4$) jest zgodne ze złożonością GLR.
W Części 7, przedstawiono empirycznie, że parsery ALL(*) dla rozpowszechnionych
języków są efektywne i wykazują w praktyce zachowanie linearne.
\par
Zalety ALL(*) wynikają z przeniesienia analizy gramatycznej do czasu parsowania,
ale wybór ten powoduje dodatkowe obciążenie na funkcjonalne testowanie.
Podobnie jak w przypadku wszystkich metod dynamicznych, programiści muszą
pokryć tyle samo pozycji gramatycznych i kombinacji sekwencji wejściowych,
aby możliwe było znalezienie niejednoznaczności gramatycznej.
Standardowe narzędzia pokrycia kodu źródłowego mogą pomóc programistom zmierzyć
pokrycie gramatyczne dla parserów ALL(*).
Wysokie pokrycie w wygenerowanym kodzie odpowiada wysokiemu pokryciu gramatycznemu.
\par
Algorytm ALL(*) jest podstawą generatora parserów ANTLR 4 (ANTLR 3 opiera się na LL(*)).
ANTLR 4 został wypuszczony w styczniu 2013 i dostaje ok. 5000 pobrań miesięcznie
(źródła, binaria, albo środowisko ANTLRworks2, licząc wejścia nie botów w
logach z unikatowych adresach IP).
Taka aktywność dostarcza dowodów, że ALL(*) jest przydatny i użyteczny.
\par
Pozostała część artykułu uwzględnia następujące tematy: Po pierwsze,
przedstawiliśmy generator parserów ANTLR 4 (Sekcja 2) i omówiliśmy strategię
analizy składniowej ALL(*) (Sekcja 3).
Następnie, określiliśmy predykatywną gramatykę, ich rozszerzone sieci przejść,
oraz lookahead DFA (Sekcja 4).
Kolejno, opisaliśmy analizę gramatyczną ALL(*) i przedstawiliśmy algorytm
analizy składniowej (Sekcja 5).
Ostatecznie, wspomagamy się twierdzeniami dotyczącymi poprawności (Sekcja 6)
i wydajności ALL(*) (Sekcja 7), oraz zbadaliśmy powiązane prace (Sekcja 8).
Załącznik A uwzględnia dowody dla twierdzenia ALL(*), Załącznik B opisuje
pragmatykę algorytmu, Załącznik C uwzględnia lewostronne rekursywne
szczegóły eliminacyjne. 
