\subsection{Raportowanie i naprawa błędów}
Predykcja ALL(*) może skanować dowolnie daleko naprzód tak że
błędna sekwencja lookahead może być długa. Domyślnie parsery
generowane przez ANTLR wypisują całą sekwencję. Do naprawy
parsery konsumują wszystkie tokeny aż będzie token
spełniający bieżącą regułę.
ANTLR dostarcza możliwośći zmiany stategii raportowania i naprawy.
\par
Parsery ANTLR emitują komunikat błędu dla błędnej frazy wejściowej
i próbują naprawić. Dla niezgodnych tokenów ANTLR
próbuje wstawiania i usuwania pojedynczych tokenów do synchronizacji.
Jeśli pozostałe symbole wejściowe są niezgodne z jakąkolwiek
produkcją bieżącego nieterminala, parser konsumuje tokeny aż znajdzie
token któy rozsądnie nastąpił by po bieżącym nieterminalu.
Następnie parser kontunuje parsowanie jakby biezący nieterminal
był prawidłowy. ANTLR naprawił obsługę błędów w porównaniu z ANTLR
3 dla podreguł EBNF przez wstawienie sprawdzania synchronizacji
na początek i do "pętli" kontunuacji test by uniknąć przedwczesnego
wyjścia z subreguły. Na przykład, rozważmy następującą regułę
definicji klasy.\\
classdef : 'class' ID 'f' member+ 'g' ;\\
member : 'int' ID ';' ;\\
Dodatkowy średnik w liście member taki jak i;; int
j; nie powinien zmuszać otaczającą regułę classdef do przerwania.
Zamiast tego, parser ignoruje dodatkowy średnik i patrzy na inny
member. Aby zredukować kaskadę błędnych komunikatów, parser
nie emituje dalszych komunikatów aż poprawnie dopasuje token.
