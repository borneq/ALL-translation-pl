\subsection{Przykładowa gramatyka}
Obrazek 1 ilustruje yacco-podobny metajęzyk ANTLR za pomocą przykładu prostego języka
programowania z przypisaniami i wyrażeniami zakończonymi średnikami. 
Istnieją dwie cechy gramatyczne, które nadają tej gramatyce własność nie-LL(*),
a zatem nie do przyjęcia dla ANTLR 3.
Po pierwsze, reguła expr jest lewostronnie rekurencyjna.
ANTLR 4 automatycznie przepisuje regułę jako nie lewostronnie rekurencyjną i jednoznaczną,
jak opisano w Sekcji 2.2. Po drugie, alternatywne produkcje reguły stat posiadają
rekurencyjny przedrostek (expr), który wystarcza aby nadać stat nierozstrzygalność
z perspektywy LL(*). ANTLR 3 wykrywa rekurencję w lewych stronach produkcji i
wykazuje niepowodzenie w decyzji z nawrotami (backtracking) podczas uruchomienia.
\par
Predykat \{!enum\textunderscore is\textunderscore keyword?\} w regule id
umożliwia lub uniemożliwia enum jako prawidłowy identyfikator według predykatu
w chwili przewidywania. Gdy predykat jest false, parser traktuje id jako
tylko id:ID; odrzucając enum jako id gdy lexer pasuje do enum jako oddzielny
symbol ID. Przykład ten przedstawia jak predykaty umożliwiają pojedynczej
gramatyce opisanie podzbiorów albo wariacji tego samego języka. 
\begin{figure}[!ht]
grammar Ex; \textcolor{PineGreen}{// generuje klasę ExParser} \\
\textcolor{PineGreen}{// akcja definiuje pole ExParser: enum\textunderscore is\textunderscore keyword} \\
@members { boolean enum\textunderscore is\textunderscore keyword = true;} \\
stat: expr \textcolor{magenta}{'='} expr \textcolor{magenta}{';'} \textcolor{PineGreen}{// produkcja 1}
\begin{adjustwidth}{1cm}{}
  | expr \textcolor{magenta}{';'} \textcolor{PineGreen}{// produkcja 2} \\
  ; 
\end{adjustwidth}
expr: expr \textcolor{magenta}{'*'} expr
\begin{adjustwidth}{1cm}{}
    | expr \textcolor{magenta}{'+'} expr \\
    | expr \textcolor{magenta}{'('} expr \textcolor{magenta}{')'} \textcolor{PineGreen}{// f(x)} \\
    | id 
    ; \\
\end{adjustwidth}    
id : ID | \{!enum\textunderscore is\textunderscore keyword?\} \textcolor{magenta}{'enum'} ; \\
ID : [A-Za-z]+ ; \textcolor{PineGreen}{// dopasowuje id z wielkimi i małymi literami} \\
WS : [ \textbackslash t \textbackslash r \textbackslash n]+ -> skip ; \textcolor{PineGreen}{// ignoruj białe znaki} \\
\caption{Przykładowa lewostronnie rekurencyjna predykatowa ANTLR 4 gramatyka Ex}
\end{figure}
