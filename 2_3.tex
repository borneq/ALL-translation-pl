\subsection{Analiza leksykalna przy użyciu ALL(*)}
ANTLR używa wariacji ALL(*) dla lekserów, które w pełni przypasowują tokeny
zamiast po prostu przewidywać produkcje, jak to robią parsery ALL(*).
Po rozbiegu, lekser tworzy DFA podobne do tego, co utworzyłyby statycznie
wyrażenia regularne w oparciu o narzędzia takie jak lex.
Podstawową różnicą jest to, że leksery ALL(*) obsługują predykatywną gramatykę
bezkontekstową a nie wyrażenia regularne, tak więc mogą rozpoznawać
bezkontekstowe symbole, takie jak zagnieżdżone komentarze i mogą wpuszczać
lub nie symbole zgodnie z semantycznym kontekstem. 
Ten zamysł jest możliwy ponieważ ALL(*) jest wystarczająco szybki, aby
obsłużyć zarówno analizę leksykalną jak i parsowanie.
\par
ALL(*) nadaje się również do analizy składniowej bez skanera ze względu na
siłę rozpoznania, która jest przydatna w leksykalnych problemach kontekstowych,
takich jak łączenie języków C i SQL.
Takie połączenie nie ma wyraźnych leksykalnych granic wytyczających leksykalne regiony: \\
 \texttt{
int next = select ID from users where name='Raj'+1; \\
int from = 1, select = 2; \\
int x = select * from; \\
}
Zobacz gramatyczny kod/dodatki/CSQL [19] dla dowiedzenia koncepcji. 
