\subsection{Detekcja konfliktu i niejednoznaczności}
Pojęcie sprzecznych konfiguracji ma kluczowe znaczenie dla analizy ALL(*).
Konflikty wywołują pracę awaryjną w pełnym przewidywaniu LL podczas przewidywania
SLL i wskazują na niejednoznaczność podczas przewidywania LL.
Wystarczającym warunkiem konfliktu między konfiguracjami jest, gdy różnią się
tylko w przewidzianej alternatywie: $(p, i, \Gamma)$ oraz $(p, j, \Gamma)$.
Wykrywanie konfliktów jest wspomagane przez dwie funkcje.
Pierwsza, \textit{getConflictSetsPerLoc} (Funkcja 8) zbiera zbiory numerów produkcji powiązanych
z wszystkimi konfiguracjami $(p, -, \Gamma)$.
Jeśli para p,$\Gamma$ przewiduje więcej niż jedną produkcję, wówczas występuje konflikt.
Poniżej przykład zbioru konfiguracji i powiązanego zbioru konfliktu:
\par
\{$\underbrace{(p,1,\Gamma),(p,2,\Gamma),(p,3,\Gamma)}_{1,2,3},
\underbrace{(p,1,\Gamma'),(p,2,\Gamma')}_{1,2},\underbrace{(r,2,\Gamma'')}_{2}$\}
\\
Te zbiory konfliktów wskazują, że lokalizacja p,$\Gamma$ jest osiągalna z produkcji {1, 2, 3},
lokalizacja p,$\Gamma'$ jest osiągalna z produkcji {1, 2},
a lokalizacja r,$\Gamma''$ jest osiągalna z produkcji {2}.
\\
\minibox[frame]{//Dla każdego p,$\Gamma$ daj zbiór alts(i) z (p,-,$\Gamma$) $\in$ konfiruracji D\\
\textbf{Function 8}:getConflictSetsPerLoc(D) \textbf{returns} zbiór zbiorów\\
s:=$\varnothing$;\\
\textbf{for} (p,-,$\Gamma$) $in$ D \textbf{do} prods:=\{i|(p,i,$\Gamma$)\}; s:=s $\cup$ prods;\\
\textbf{return} s;
}
\par
Druga funkcja, \textit{getProdSetsPerState} (Funkcja 9), jest podobna, ale zbiera numery
produkcji powiązane tylko ze stanem p ATN.
Dla takiego samego zbioru konfiguracji, \textit{getProdSetsPerState} oblicza następujące zbory konfliktu:
\par
\{$\underbrace{(p,1,\Gamma),(p,2,\Gamma),(p,3,\Gamma),(p,1,\Gamma'),(p,2,\Gamma')}_{1,2,3},
\underbrace{(r,2,\Gamma'')}_{2}$\}
\par
Wystarczający warunek dla niepowodzenia w przewidywaniu LL (\textit{LLpredict})
z SLL mógłby wystąpić w przypadku, gdy jest przynajmniej jeden zbiór sprzecznych konfiguracji:
\textit{getConflictSetsPerLoc} zwraca co najmniej jeden zbiór z więcej niż jedną liczbą produkcji.
Np. konfiguracje $(p, i, \Gamma)$ i $(p, j, \Gamma)$ występują w parametrze D.
Jednak naszym celem jest kontynuowanie przy użyciu przewidywania SLL tak długo,
jak to tylko możliwe, ponieważ przewidywanie SLL aktualizuje cache lookahead DFA.
W tym celu, przewidywanie SLL postępuje, jeśli występuje przynajmniej jedna konfiguracja
niesprzeczna (gdy \textit{getProdSetsPerState} zwraca co najmniej jeden zbiór wielkości 1).
Mając nadzieję, że więcej lookahead doprowadzi do zbioru konfiguracji,
który przewiduje jednoznaczną produkcję za pośrednictwem niesprzecznych konfiguracji.
Na przykład, decyzja dla $S \rightarrow a|a|a \cdot_p b$ jest niejednoznaczna względem $a$
pomiędzy produkcjami 1 i 2, jednakże jest jednoznaczna względem ab.
(Lokalizacja $\cdot_p$ jest stanem ATN między a i b.)
Po dopasowaniu danych wejściowych a,
zbiór konfiguracji wyglądałby następująco \{$(p'_S, 1, []), (p'_S, 2, []), (p, 3, [])$\}.
Funkcja \textit{getConflictSetsPerLoc} zwraca \{\{1, 2\}, \{3\}\}.
Następnie \textit{move-closure} względem $b$ prowadzi do zbioru konfiguracji
niesprzecznych \{($p'_S$, 3, [])\} z (p, 3, []), omijając konflikt.
W przypadku gdy wszystkie zwrócone zbiory z \textit{getConflictSetsPerLoc} przewidują więcej
niż jedną alternatywę, żadna ilość lookahead nie doprowadzi do jednoznacznego przewidywania.
Analiza musi zostać wznowiona ponownie ze stosem wywołań $\gamma_0$ za pośrednictwem LLpredict.
\\
\minibox[frame]{//Dla każdego p daj zbiór alts(i) z (p,-,-) $\in$ konfiruracji D\\
\textbf{Function 9}:getProdSetsPerState(D) \textbf{returns} zbiór zbiorów\\
s:=$\varnothing$;\\
\textbf{for} (p,-,-) $in$ D \textbf{do} prods:=\{i|(p,i,-)\}; s:=s $\cup$ prods;\\
\textbf{return} s;
}
\par
Konflikty podczas symulacji LL są niejednoznacznościami i występują gdy każdy
zbiór konfliktu z \textit{getConflictSetsPerLoc} zawiera więcej niż 1 produkcję – każda lokalizacja
w D jest osiągalna z więcej niż 1 produkcją.
Gdy tylko wiele subparserów osiągnie to samo $(p, -, \Gamma)$ wszystkie przyszłe
symulacje wyprowadzone z $(p, -, \Gamma)$ będą zachowywać się identycznie.
Więcej lookahead nie rozwiąże niejednoznaczności.
Przewidywanie może zostać przerwane w tym momencie i zaraportować lookahead przedrostka
$u$ jako niejednoznaczne, jednakże LLpredict nadal postępuje aż się upewni
\textbf{która} produkcja $u$ jest niejednoznaczna.
Należy wziąć pod uwagę zbiory konfliktu \{1,2,3\} oraz \{2,3\}.
Ponieważ oba mają więcej niż jeden stopień, zbiory te reprezentują niejednoznaczność,
jednakże dodatkowe dane wejściowe będą identyfikować czy $u \preceq w_r$ jest niejednoznaczne
względem \{1,2,3\} czy \{2,3\}.
Funkcja \textit{LLpredict} postępuje aż wszystkie zbiory konfliktu, które identyfikują niejasności
będą równe; warunek x = y oraz |x| > 1  $\forall x,y \in altsets$ odzwierciedla ten test.
\par
W celu wykrycia konfliktów, algorytm porównuje grafowo-strukturalne stosy często.
Technicznie konflikt występuje gdy konfiguracje $(p, i, \Gamma)$ i $(p, j, \Gamma')$
występują w tym samym zbiorze konfiguracji z $i \neq j$ oraz przynajmniej jednym
śladem stosu zarówno dla $\Gamma$ jak i $\Gamma'$.
Z faktu iż weryfikacja pod kątem przecięć grafu jest kosztowna, algorytm wykorzystuje
regułę równości, $\Gamma = \Gamma'$, jako metodę heurystyczną.
Równość jest znacznie szybsza ze względu na wspólne podgrafy.
Wykres algorytmu równości często może sprawdzać identyczność węzła do porównywania dwóch
całych podgrafów. W najgorszym przypadku, równość kontra podzbiór heurystycznie opóźnia
wykrycie konfliktu aż GSS między sprzecznymi konfiguracjami będzie prostymi liniowymi stosami,
gdzie przecięcie grafów jest takie same jak graf równości.
Kosztem tej heurystyki jest głębszy lookahead.

