\section{Algorytm analizy składniowej ALL(*)}
Z definicji gramatyki, sformalizowanych ATN i lookahead DFA, możemy przedstawić główne funkcje
algorytmu analizy składniowej ALL(*).
W części tej podsumowano wszystkie funkcje i opisano jak łączą się one, następnie omówiono
wykres struktury danych przed przedstawianiem samych funkcji. Ostatecznie przedstawiono
przykład jak działa algorytm.
\par
Parsowanie rozpoczyna się od funkcji \textit{parse}, która zachowuje się jak konwencjonalna
zstępująca analiza funkcji LL(k), z wyjątkiem tego że parsery ALL(*) przewidują
produkcje ze specjalną funkcją zwaną \textit{adaptivePredict},
zamiast zwykłego mechanizmu „przełącz na następny typ symbolu k”.
Funkcja \textit{adaptivePredict} symuluje ATN odzwierciedlając pierwotną gramatykę
predykatu w wyborze produkcji $\alpha_i$ w celu rozwinięcia punktu decyzyjnego
$A \rightarrow \alpha_1 | ... | \alpha_n$.
Od strony koncepcyjnej, przewidywanie jest funkcją bieżącej analizy składniowej stosu wywołań
$\gamma_0$, pozostałych danych wejściowych $w_r$ i stanu użytkownika $\mathbb{S}$ jeśli A posiada predykaty.
Jeżeli chodzi o wydajność, przewidywanie ignoruje $\gamma_0$, gdy to tylko możliwe (Sekcja 3.1)
i używa minimalnego lookahead z $w_r$.
\par
Aby uniknąć powtarzania symulacji ATN dla tych samych danych wejściowych i symboli nieterminalnych,
\textit{adaptivePredict} składa DFA, który gromadzi mapowania danych wejściowych do przewidywanej produkcji,
jeden DFA na symbol nieterminalny.
Należy mieć na uwadze, że każdy stan DFA, D, jest zbiorem konfiguracji ATN dopuszczalnym po dopasowaniu
symboli lookahead prowadzących do tego stanu.
Funkcja \textit{adaptivePredict} wywołuje \textit{startState} do utworzenia początkowego stanu DFA,
\( D_0 \) a następnie wywołuje \textit{SLLpredict} w celu uruchomienia symulacji.
\par
Funkcja \textit{SLLpredict} dodaje ścieżki do lookahead DFA, które pasują do niektórych lub wszystkich
$w_r$ poprzez wielokrotne wołanie \textit{target}. Funkcja \textit{target} oblicza DFA docelowego stanu D'
z bieżącego stanu D przy użyciu operacji \textit{move} i \textit{closure} podobnych do tych
znalezionych w \textit{subset construction}.
Funkcja \textit{move} wykrywa wszystkie konfiguracje ATN jako osiągalne na bieżącym symbolu wejściowym,
a funkcja \textit{closure} wykrywa wszystkie konfiguracje jako osiągalne bez przemierzenia krawędzi terminalnej.
Podstawową różnicą od podzbioru konstrukcji jest to, że \textit{closure} symuluje wywołanie
i powrót podautomatów ATN związanych z symbolami nieterminali.
\par
W przypadku, gdy symulacja SLL wykryje konflikt (Sekcja 5.3), \textit{SLLpredict} przewija dane wejściowe
i wywołuje \textit{LLpredict} w celu ponowienia próby przewidywania, tym razem biorąc pod uwagę $\gamma_0$.
Funkcja \textit{LLpredict} jest podobna do \textit{SLLpredict}, ale nie aktualizuje symboli nieterminalnych
DFA, ponieważ DFA musi być niezależny od stosu, aby mógł być wykorzystywany we wszystkich kontekstach stosu.
Konflikty w zbiorach konfiguracji ATN wykryte przez \textit{LLpredict} reprezentują niejednoznaczności.
Obie funkcje przewidywania używają \textit{getConflictSetsPerLoc} do wykrywania konfliktów,
które są konfiguracjami określającymi tą samą lokalizację parsera, ale różne produkcje.
Aby uniknąć niepowodzenia \textit{LLpredict}, \textit{SLLpredict} wykorzystuje \textit{getProdSetsPerState}
aby zobaczyć czy potencjalnie bezkonfliktowa ścieżka DFA pozostaje gdy \textit{getConflictSetsPerLoc}
zgłasza konflikt.
Jeśli tak, to warto kontynuować proces \textit{SLLpredict}, mając nadzieję że więcej lookahead
rozwiąże konflikt bez uciekania się do pełnej analizy składniowej LL.
\par
Zanim szczegółowo omówimy te funkcje, przedstawimy przegląd fundamentalnych danych struktury grafu,
które efektywnie zarządzają wieloma stosami wywołań GLL i GLR.
