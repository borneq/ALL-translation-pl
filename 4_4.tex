\subsection{Lookahead DFA}
Parsery ALL(*) odnotowują wyniki przewidywań uzyskane z symulacji ATN wraz z lookahead DFA,
które są rozszerzonym DFA z akceptowalnymi stanami, odzwierciedlającymi numer przewidywanej produkcji.
Występuje jeden akceptowalny stan decyzji na produkcję.
Definicja 4.1. Lookahead DFA to DFA rozszerzony o akceptujące stany,
który dostarcza numer przewidywanej produkcji.
Dla gramatyki predykatowej \(G = (N, T, P, S, \Pi, \mathcal{M}), DFA M = (Q, \Sigma, \Delta, D_0, F) \) gdzie
\begin{itemize}
\item Q jest zbiorem stanów
\item \(\Sigma = T \)jest alfabetem krawędzi
\item \( \Delta \) jest funkcją przejścia mapującą \( Q \times \Sigma \rightarrow Q \)
\item \( D_0 \in Q \) jest stanem startowym
\item \( F \subset Q  = \{  f_1, f_2, ..., f_n\}\) stany końcowe, jeden \( f_i \)  dla produkcji i
\end{itemize}
Przejście w \( \Delta \) ze stanu p do stanu q na symbolu \( a \in \Sigma \)
ma postać \( p \overset{a}{\rightarrow} q \) i wymagamy by \(p \overset{a}{\rightarrow} q' \)
implikowało  q = q'.

