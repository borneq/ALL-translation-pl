\section{Powiązane prace}
Przez dekady pracowano w kierunku zwiększenia
siły rozpoznawania recognition wydajnych ale nie ogólnych
parserów LL i LR i zwiększenia wydajności ogólnych algorytmów
Earleya O(n3) [8]. Parr [21] i Charles [4] statystycznie generowane parsery LL(k) i LR(k)
dla k > 1. Parr i Fisher’s LL(*) [20] i Bermudez i Schimpf’s LAR(m) [2]
statystycznie liczone parsery LL i LR wzbogacone o cykliczny
DFA który mógł badać dowolną ilość podglądanych symboli. Te
parsery bazowały na parserach LL-regular [12] i LR-regular [7]
które miały niepożądaną właściwość bycia nierozstrzygalnym w ogólności.
Wprowadzenia algorytmu z nawrotami(backtrackingu) dramatycznie zwiększyło się rozpoznawania
i uniknięto kwestii nierozstrzygalności statycznej analizy gramatyki
ale jest niepożądane ponieważ osadzone mutatory redukowały wydajność
i komplikowały debugowanie krok po kroku. parsery Packrat (PEGs) [9] próbowały decyzji
produkcji w kolejności I wybierały pierwszą która dawała sukces. PEG są
O(n) ponieważ one pamiętają częściowy rezultat parsowania ale cierpią
na aberrację a | ab gdzie ab jest milcząco nie brane pod uwagę.
\par
Aby zwiększyć ogólną wydajność parsowania, Tomita [26] wprowadził GLR, ogólny algorytm
bazujący na LR(k) który konceptualnie forkuje subparsery dla każdego stanu
LR(k) z konfliktem w czasie parsowania to badania wszystkich możliwych ścieżek.
Tomita pokazuje że GLR jest 5x-10x szybszy niż Earley. Kluczowy komponent
parsowania GLR jest graph-structured stack (GSS) [26] który zapobiega parsowaniu
tego samego wejścia dwukrotnie w ten sam sposób.
(GLR wkłada symbole wejściowe i stany LR na GSS podczas gdy ALL(*) wkłada stany ATN)
Elkhound [18] wprowadził hybrydowe parsery GLR które używają pojedynczy stos
dla wszystkich decyzji LR(1) a GSS kiedy niezbędne jest dopasowanie
niejednoznacznych porcji wejścia. (Zobaczyliśmy że parsery Elkhounda
są szybsze niż inne GLR.) GLL [25] jest analogiem LL parserów GLR
i również używa subparserów i GSS do badania wszystkich możliwych ścieżek;
GLL używa lookahead k = 1 gdzie można dla wydajności.
GLL jest O(n3) a GLR jest O(np+1)
gdzie p jest długością najdłuższej produkcji gramatyczne.
\par
Parsery Earleya skalują się dobrze od  O(n) dla deterministycznych gramatyk
do O(n3) dla niejednoznacznych gramatyk ale wydajność
nie jest dość dobra w ogólnym użyciu.
LR(k) maszyny stanów mogą polepszyć wydajność takich parserów
prze statystyczne obliczanie tak dużo jak to możliwe. LRE [16] jest jednym
takim przypadkiem. Mimo tych optymizacji ogólne algorytmy pozostają
bardzo wolne w porównaniu do deterministycznych parserów wzbogaconych
o głęboki podgląd symboli.
\par
Problem z dowolną liczbą podglądanych symboli jest taki że jest niemożliwe
wyliczyć statystycznie dla wielu użytecznych gramatyk (warunek LLregular jest nierozstrzygalny.) Przez przeniesienie analizy podglądanych symboli do czasu analizy ALL(*) zdobył siłę do obsługi jakiejkolwiek gramatyki bez lewostronnej rekurencji ponieważ może uruchomić subparsery do określenia jaka ścieżka prowadzi to poprawnego parsowania. Inaczej niż GLR,
spekulacja zatrzymuje się kiedy wszystkie pozostałe subparsery są powiązane z pojedynczą alternatywą a produkcji, tak więc wyliczając minimalną sekwencję podglądanych symboli.. Aby zyskać wydajność, ALL(*) zapamiętuje mapowanie z tej sekwencji lookahead to przewidzianej produkcji używając DFA dla użycia przez kolejne decyzje.
Podzbiory języka bezkontekstowego napotykane podczas parsowania są skończone i dlatego języki ALL(*) lookahead są regularne.
Ancona et al [1] również wykonują analizę czasu parsowania ale oni
jedynie obliczają stały lookahead LR(k) i nie adaptują do faktycznego wejścia jak to robi ALL(*).
Perlin [23] działa na RTN jak ALL(*) ale oblicza  lookahead k = 1 podczas parsowania.
\par
ALL(*) jest podobne do Earley w tym że obydwa są top-down i
operują na reprezentacji gramatyki w czasie parsowania ale
Earley parsuje nie licząc lookahead DFA. W tym sensie,
Earley nie wykonuje analizy gramatycznej, Earley również nie zarządza jawnie
GSS podczas parsowania. Zamiast tego element w stanach Earleya
mają “wskaźniki rodzica” które odnoszą się do innych stanów
kiedy są w wątku razem
Operacja SCANNER Earleya
koresponduje do funkcji ALL(*) move. Operacje PREDICTOR i COMPLETER
korespondują do push i pop in funkcji closure ALL(*)
Stan Earleya jest zbiorem wszystkich konfiguracji parsera osiągalnych
w pewnej absolutnej głębokości wejścia podczas gdy stan DFA ALL(*) jest zbiorem konfiguracji
osiągalnych z głębokości podglądu relatywnej do bieżącej decyzji
Inaczej niż w kompletnie ogólnych algorytmach, ALL(*) wyszukuje
pojedyncze parsowanie wejścia, które pozwala użyć wydajnego stosu LL podczas parsowania.
\par
Strategie parsowania które ciągle spekulują lub wspomagają niejednoznaczność
mają trudności z mutatorami ponieważ ciężko wtedy o cofnięcie działania.
Brak mutatorów redukuje ogólność semantycznych predykatów które zmieniają parsowanie ponieważ nie można testować dowolnie stanu obliczonego wcześniej podczas parsowania.
Rats! [10] wspomaga ścisłe semantyczne predykaty a Yakker [13] wspomaga
semantyczne predykaty które są funkcjami poprzednio parsowanych symboli terminalnych.
Ponieważ ALL(*) nie spekuluje podczas faktycznego parsowania, wspomaga dowolne mutatory i semantyczne predykaty..
Miejsce wyklucza bardziej szczegółowe dyskusje o powiązanych pracach; bardziej szczegółowa analiza może być znaleziona w [20].
