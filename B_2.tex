\subsection{Wyliczenie semantycznych predykatów}
Dla jasności, algorytm opisany w tym artykule uźywa czystej
symulacji ATN dla wszystkich decyzji które mają that semantyczne predykaty
na lewymch krawędziach produkcji W praktyce ANTLR używa lookahead
DFA które śledzą predykaty i akceptują stany do obsługi senemtycznej
zależnej od kontekstu predykcji. Śledzenie predykatów w
DFA pozwala predykcjom uniknąć drogiej symulacji ATN jeśli
wyliczenie predykatu w ciągu symulacji SLL przewidzi unikalną
produkcję. Semantyczne predykaty nie są powszechne ale są krytyczne
do rozwiązywania pewnych zależnych od kontekstu problemów paroswania, to jest,
predykaty są używane wewnętrznie przez ANTLR do kodowania
priorytetu operatorów przy przepisywaniu lewostronnie rekurencyjnych reguł.
Tak że to jest warte dodatkowej złożoności do obliczenia
predykatów podczas predykcji SLL.
Rozważmy repdykowaną regułę z Części 2.1:
\par
id : ID | \{!enum_is_keyword\}? 'enum' ;
\\
Druga produkcja jest wykonalna tylko kiedy !enum_is_keyword
jest wylcizony do true. W skrócie to oznacza że parser mógłby
potrzebować dwa two lookahead DFA, jeden na warunke semantyczny.
Zamiast tego implementacja ANTLR ALL(*) tworzy DFA (przez
predykcję SLL) z krawędzią $D_0 \overset{enum}{\longrightarrow} f_2$
gdzie $f_2$ jest wzbogaconym stanem akceptującym DFA
który testuje !enum_is_keyword.
Funkcja adaptivePredict zwraca produkcję 2 na enum jeśli
!enum_is_keyword w przeciwnym razie rzuca wyjątek no-viable-alternative.
\par
Algorytm opisany w tym artykule również nie obsługuje
semantycznych predykatów poza regułą wejścia decyzji.
W praktyce analiza ALL(*) musi wyliczyć wszystkie predykaty osiągalne
z reguły wejścia decyzji bez przestępowania krawędzi terminalu
w ATN. Na przykład uproszczony algorytm ALL(*) w tym artykule rozważa jedynie
predykaty $\pi_1$ i $\pi_1$ dla produkcji S dla następującej (niejednoznacznej) gramatyki.
\begin{itemize}
\item $S\rightarrow \{\pi_1\}? Ab | \{\pi_2\}? Ab$
\item $A\rightarrow \{\pi_3\}? a | \{\pi_4\}? a$
\end{itemize}
Wejście dopasowuje zarówno alternatywę S i w praktyce
ANTRL wylicza "$\pi_1$ and ($\pi_3$ or $\pi_4$)" do testowania wykonalności
pierwszej procukcji a nie tylko "%\pi_1%. Po symulacji podmaszyn S
i A, lookahead DFA dla S byłby
$D_0\overset{a}{\rightarrow}D'\overset{b}{\rightarrow}f_{1,2}$
Wzbogacony akceptujący stan $f_{1,2}$ przewiduje produkcje
1 lub 2 w zależności od semantycznego kontekstu odpowiednio
$\pi_1 \wedge (\pi_3 \vee \pi_4)$
Do trzymania śladu kontekstu semantycznego podczas symulacji SLL,
konfiguracje ANTLR ATN
zawierają dodatkowy $\pi$: (p, i, $\Gamma$, $\pi$).
Element $\pi$ niesie ze sobą
kontekst semantyczny i ANTLR pamięta pary predikat-do-produkcji
we wzbogaconym stanie akceptującym DFA.