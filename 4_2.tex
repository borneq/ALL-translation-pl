\subsection{Sprecyzowanie niejednoznaczności}
Niejednoznaczna gramatyka jest taką, w który taka sama sekwencja danych wejściowych
może być rozpoznawana na wiele sposobów.
Zasady na Rysunku 6 nie wykluczają niejednoznaczności. 

\begin{figure}[h]
\noindent\rule{\linewidth}{1pt} \\
\( A \in N \) symbol nieterminalny \\
\( a,b,c,d \in T \) symbol terminalny \\
\( X \in (N \cup T) \) element produkcji \\
\( \alpha,\beta,\delta \in X^* \) sekwencja symboli gramatyki \\
\( u,v,w,x,y \in T^* \) sekwencja symboli terminalnych \\
\( \varepsilon \) pusty ciąg \\
\$ koniec strumienia symboli \\
\( \pi \in \Pi \) predykat w języku docelowym \\
\( \mu \in \mathcal{M} \) akcja w języku docelowym \\
\( \lambda \in (N \cup \Pi \cup \mathcal{M}) \) etykieta redukcji \\
\( \overset{\rightarrow}{\lambda} = \lambda_1 ..\lambda_n\) sekwencja etykiet redukcji \\
Reguły produkcji:\\
\( A \rightarrow \alpha_i \) i-ta bezkontekstowa produkcja A \\
\( A \rightarrow \{ \pi_i \} ? \alpha_i \) i-ta produkcja predykowana na semantyce \\
\( A \rightarrow \{ \mu_i \} \) i-ta produkcja z mutatorem \\
\noindent\rule{\linewidth}{1pt} 
\caption{Notacja gramatyki predykatowej}
\end{figure}

\begin{figure}[h]
\noindent\rule{\linewidth}{1pt} \\
\begin{adjustwidth}{2cm}{}
Prod \( \frac{A \rightarrow \alpha}{ (\mathbb{S}, uA\delta) \Rightarrow (\mathbb{S}, u\alpha\delta)} \) 
\\
\( \pi(\mathbb{S}) \)
\\
Sem \( \frac{A \rightarrow \{ \pi \} ? \alpha}{ (\mathbb{S}, uA\delta) \Rightarrow (\mathbb{S}, u\alpha\delta)} \) 
\\
Action \( \frac{A \rightarrow \{ \mu \} }{ (\mathbb{S}, uA\delta) \Rightarrow (\mu (\mathbb{S}), u\delta)} \) 
\\
Closure \( \frac{ (\mathbb{S},\alpha) \Rightarrow (\mathbb{S'},\alpha'),
(\mathbb{S'},\alpha') \Rightarrow^* (\mathbb{S''},\beta) }{(\mathbb{S},\alpha) \Rightarrow^* (\mathbb{S''},\beta)} \) 
\\
\end{adjustwidth}
\noindent\rule{\linewidth}{1pt}
\caption{Reguły lewostronnego wyprowadzania gramatyki predykatowej}
\end{figure}

Jednakże, dla praktycznego analizatora języka programowania,
wszystkie dane wejściowe powinny odpowiadać analizie składniowej.
W tym celu, ALL(*) używa kolejności produkcyjnych w regule do rozpoznania
niejednoznaczności na korzyść produkcji o najniższym numerze.
Strategia ta jest zwięzłym sposobem dla programistów, aby rozpoznać niejednoznaczności
automatycznie oraz znaną niejednoznaczność if-then-else w zwykły sposób
poprzez kojarzenie ‘else’ z ostatnim ‘if’. Parsery PEG i Bison mają te same zasady rozpoznawania.
\par
Do rozpoznania niejednoznaczności, które zależą od bieżącego stanu S, programiści mogą
wprowadzać semantyczne predykaty, jednakże muszą wzajemnie wykluczyć dla wszystkich
potencjalnie niejednoznacznych sekwencji wejściowych do otrzymania tego typu jednoznacznych produkcji.
Statycznie wzajemnych wykluczeń nie da się wymusić, ponieważ predykaty są napisane w języku Turinga.
Jednakże, w czasie analizy ALL(*) wylicza predykaty i dynamicznie raportuje
frazy wejściowe dla których wielokrotne, predykowalne produkcje są wykonalne.
Jeśli programistom nie udaje się ustalić wzajemnej wyłączności, ALL(*)
wykorzystuje kolejność produkcji do rozpoznania niejednoznaczności.
