\subsection{Gramatyka predykatowa}
Aby sformalizować parsing ALL(*), najpierw należy formalnie
określić gramatykę predykatową z której jest ona wyprowadzana.
Gramatyka predykatowa \(G = (N, T, P, S, \Pi, \mathcal{M})\) składa się z następujących elementów: 
\begin{itemize}
\item N jest zbiorem nieterminali (nazwy reguł)
\item T jest zbiorem terminali (symboli) 
\item P jest zbiorem produkcji 
\item \(S \in N\) jest symbolem początkowym 
\item \( \Pi \) jest zbiorem efektów predykatów wolnych od efektów ubocznych
\item \(\mathcal{M}\) jest zbiorem działań (mutatorów) 
\end{itemize}
Gramatyki predykatowe ALL(*) różnią się od tych z LL(*) [20] jedynie tym że
gramatyka ALL(*) nie potrzebuje albo nie obsługuje składni predykatów.
Gramatyki predykatowe w formalnych częściach niniejszego dokumentu używają
notacji przedstawionej na Rysunku 5.
Zasady wyprowadzania na Rysunku 6 określają znaczenie gramatyki predykatowej.
Do obsługi semantycznych predykatów i mutatorów, zasady odnoszą się do stanu S,
który abstrahuje stan użytkownika podczas analizowania.
Przejście \(( \mathbb{S}, \alpha) \Rightarrow ( \mathbb{S'}, \beta) \)
może zostać odczytane: „W stanie S maszyny, sekwencja gramatyczna \(\alpha\)
redukuje się w jednym kroku do zmodyfikowanego stanu S' i sekwencji gramatycznej \(\beta\)”.
Przejście \((\mathbb{S}, \alpha) \Rightarrow^* (\mathbb{S'}, \beta)\)
oznacza wielokrotne zastosowanie jednostopniowej zasady redukcji.
Te zasady redukcji określają wyprowadzenie lewostronne.
Produkcja z semantycznym predykatem \(\pi_i\) jest wykonalna tylko wtedy,
gdy \(\pi_i\) jest true z bieżącym stanem \(\mathbb{S}\).
Na koniec akcja produkcji używa określonego mutatora \(\mu_i\) aby zaktualizować stan.
\par
Formalnie, język generowany przez sekwencję gramatyczną \(\alpha\)
w stanie użytkownika \(\mathbb{S}\)
jest \(L(\mathbb{S},\alpha) = \{w| (\mathbb{S},\alpha)\Rightarrow^*(\mathbb{S'},w)\} \)
a język gramatyki G jest \(L(\mathbb{S}_0,G) = \{w| (\mathbb{S}_0,S)\Rightarrow^*(\mathbb{S},w)\} \)
dla początkowego stanu \(\mathbb{S}_0\) użytkownika (\(\mathbb{S}_0\) może być puste).
Jeśli \(\mu\) jest przedrostkiem \(w\) albo równe \(w\), zapisujemy \(\mu \preceq w\).
Język L jest ALL(*) wtedy i tylko wtedy jeśli występuje gramatyka ALL(*) dla L.
Teoretycznie, klasa języka L(G) jest rekurencyjnie przeliczalna,
ponieważ każdy mutator może być maszyną Turinga.
W rzeczywistości, autorzy gramatyki nie wykorzystują tego ogólnika,
tak więc jest to przyjęty sposób postępowania do rozważenia klasy języka jako kontekstowego.
Klasa jest zależna od kontekstu, a nie bezkontekstowa,
ponieważ predykaty mogą badać stos wywołań i terminale z lewej i prawej.
\par
Ten formalizm uwzględnia różne składniowe ograniczenia, które nie są obecne w rzeczywistej
gramatyce ANTLR, na przykład zmuszając mutatory do ich własnych zasad i wykluczając
powszechne notacje Rozszerzonego (Extended) BNF (EBNF),
takie jak domknięcia \(\alpha^*\) i \(\alpha^+\).
Możemy ustalić te ograniczenia bez utraty ogólnika, ponieważ jakakolwiek
gramatyka w ogólnej formie może przekładać się na bardziej ograniczoną formę. 
